\documentclass[12pt, letterpaper]{article}

\usepackage[T1]{fontenc}
\usepackage[utf8]{inputenc}
\usepackage{lmodern}
\usepackage[hidelinks]{hyperref}
\usepackage[T1]{fontenc}
\usepackage[utf8]{inputenc}
\usepackage{lmodern}

\usepackage[english]{babel}
\usepackage{csquotes}

\usepackage[notes,backend=biber]{biblatex-chicago}

% REMINDER for myself:  The following characters are ``special''
%
%         #  $  %  &  _  {  }  ~  ^  \
%
% So remember to escape them


% ***************************************************************************
% Page style and formatting options
% ***************************************************************************

				% This is just to make the margins
				% look nice for 8.5x11 paper
				% 1 inch border on left, right, and top
				% and 1 inch between bottom of text and
				% bottom of paper, but .5 inch between
				% bottom of footer and bottom of paper.

% Preamble

% ***************************************************************************
% Page style and formatting options
% ***************************************************************************

% This is just to make the margins
% look nice for 8.5x11 paper
% 1 inch border on left, rigth, and top
% and 1 inch between bottom of text and
% bottom of paper, but .5 inch between
% bottom of footer and bottom of paper.

% \setlength{\textwidth}{6.0in}
% \setlength{\oddsidemargin}{0in}
% \setlength{\evensidemargin}{0in}
% \setlength{\topmargin}{-0.5in}
% \setlength{\headheight}{0.25in}
% \setlength{\headsep}{0.25in}
% \setlength{\textheight}{8.5in}
% \setlength{\footskip}{20pt}

% \setlength{\topskip}{0in}	% No blank space on first page

\setcounter{secnumdepth}{0}	% Turn of section numbering

\setlength{\parindent}{0in}	% I hate indented paragraphs
\setlength{\parskip}{0.1in}

\title{Finding Friendship in the Absence of Universal Values}
\author{Sylvan Martin}
\date{October 15th, 2021}

\bibliography{bibliography}

% Body

\begin{document}

\maketitle
\thispagestyle{empty}
\newpage

How can two people form a meaningful friendship when modern society proposes no coherent system of common values? 
Cicero, a Roman statesman and philosopher, argues that friendship is founded on virtue, but he lived in a 
society where everyone could agree to a strict moral code, and could agree on what that virtue was. Are we 
doomed to never have any significant friendship now that there is no such agreed upon concept of virtue? 
Cicero was right that friendship is based on virtue, but he was lucky enough to live in a society where that 
virtue was already known. If virtuous friendships are to be possible in modern times, they cannot be based 
initially on such a generally shared concept as virtue, since none currently exists. Where then, can they 
find their firm foundation?

Cicero claims that friendship is between good people, that only a wise man is good, and that a wise man lives 
in accord with nature.\autocite[sec. 18]{cicero} We will work our way up this chain of dependency to the concept of friendship.

First, what is nature? What does it mean to live in accord with it? Although there were Roman religions as 
well as Greek Hellenism, Cicero’s philosophy is primarily secular, meaning he will avoid the supernatural when
discussing his thoughts. His idea of nature, however, is possibly 
the closest he gets to talking about a spiritual entity. While Cicero never fully defines his idea of natural 
law (at least in the context of friendship) in the same way he has a long discussion about friendship, he frequently comments on nature’s effects, 
going so far as to say that “our feelings of love and affection for those we judge to be good people spring 
from nature itself.”\autocite[sec. 32]{cicero} Natural law, then, is the driving force that humans tend to obey, and contradictory to 
how we think of nature now, it is what sets us apart from beasts. Nature also gives us the first ingredient to any kind of 
friendship: our setting. Cicero believes “we were created in such a way so that there is a bond between us 
all that grows stronger the closer we are to each other,'' and that “nature herself has produced a friendship 
of sorts,”\autocite[sec. 19]{cicero} meaning we are born into a certain group of people we find naturally favorable. In this light, 
nature becomes the beginning of the field of psychology that is used to justify certain human desires or actions 
before psychology (as we know it) fully develops. It is because all human society will follow natural law that 
friendship can even come to fruition.

This brings us to the next item on the chain, wisdom. According to Cicero, a wise man lives in accord with 
nature.\autocite[sec. 32]{cicero} This may seem like too easy a trait! After all, if men tend to follow natural law as it is what is 
``natural,'' then does that mean that wisdom just comes naturally? Not quite. To live in accord with nature 
is not just to follow all human desires and do whatever seems natural. In Greco-Roman society, wisdom 
could also be understood as prudence, or foresight. Therefore, to live in accord with nature does not mean 
to chase all bodily desires, it instead means to be \emph{aware} of them and discipline them by an act of will in accord with the natural law.
Because only a wise man is good, it follows that Cicero’s idea of morality is a right alignment with 
nature. Essentially, natural law is his basis for ethics.

Moving up the chain of reasoning, Cicero defines friendship as ``nothing other than agreement with goodwill 
and affection between people about all things divine and human.''\autocite[sec. 20]{cicero} Essentially, it is an agreement between two 
people following natural law. This works because Roman society had a very common idea of what natural law even 
was. With strict family values, loyalty to the state, and the patronage system, 
it might seem obvious to Cicero, a man who serves the state, that this idea of natural law is universal. 
After all, the Roman empire at its peak might as well have been considered universal in their eyes. It is 
only with this strict fabric that Cicero can then make the conclusion ``friendship is founded on virtue.''\autocite[sec. 37]{cicero} 
Of course this statement comes naturally to a Roman who already agrees on the idea of virtue with every other 
high-class Roman listening.

But is this even applicable to modern life? Does such a common idea of virtue even exist anymore? First let’s 
examine this in the context of America. One of the strengths of American society is its diversity and lack of 
homogeneity. Claimed to be the “melting pot of cultures,” there is no one single American culture. Even years
of rule by Judeo-Christian ethics would not stop the immigration that results in modern day America. One would be hard 
pressed to even find one state or county of people that could nearly universally agree on what virtue was. This 
partly exposes one of the faults in Cicero’s reasoning. 
Cicero concluded that friendship was natural because his he understood to universal natural law to be, for lack of a better word, universal.
However, such “universality” does not exist in America. 

Wait! Even if it doesn’t exist in America, perhaps in a country with a rich, ancient culture and history, 
virtue is already commonplace and agreed upon by every individual. Take for example China. One of the world’s 
oldest civilizations, one of China’s strengths has been its commitment to Confucianism, even in some sense to 
the modern day. According to Amita Sathe, a writer for \textit{Economic \& Political Weekly} concerned with 
the modernization of China, this common value of Confucianism lead to not only ``homogeneity, but also\ldots
a continuous 1,000 year-old history, to obey\ldots without questioning.''\autocite[p. 1441]{china_thing} Although it may 
not be fair to label the entire culture as never questioning orders, it is clear that the impact of 
a common framework leads to a vastly different society than our liberal America. Sathe even says that homogeneity
is part of the Chinese psyche.\autocite[p. 1439]{china_thing}. This has its strengths! Such a strict common agreement on what is right is clearly one of the reasons why sthe Chinese 
civilization has lasted so long, and is arguably one of the only ancient civilizations that has continuously 
lasted to this day. However, even a country as homogenous as China is having a crisis of friendship. Does this 
shatter Cicero’s entire concept of friendship as being based on common values? Not at all. China is especially 
unique in that it has rapidly urbanized\autocite[p. 1439]{china_thing} faster than virtually any other civilization in history. Naturally, an 
enormous change in habitat that a civilization may not be ready for can result in friendship being harder to 
come by, and even one decribed as ``bleak''.\autocite[p. 1439]{china_thing} After all, Cicero argues that nature places us in relation to others to form friendships, but if that 
position is in an enormous city where all social interactions are fleeting, we seem destined to be hard pressed 
to make friends.

Returning to the case of America, we still are lacking a common moral foundation to stand as any sort of natural 
law. Does this doom friendship in any diverse society? In Cicero’s time, virtue was a known concept that was 
built into a rigid social fabric, and upon that fabric friendships are built. Now, the reverse is true. 
Through the relationships and friendships we form, we can see what virtues we each value, in a time where 
there is not such a rigid social fabric that we can all agree to. 

When we go about our daily lives, interacting with people and making light conversations, we catch a glimpse of 
their core values, even if they are never explicitly stated. Every so often, we hear in conversation that 
someone else’s core values may align with ours. This is what subconsciously prompts a desire to be friends. 
It is a near universal experience to just “click” with another person. This would be what Cicero described 
as an ``agreement with goodwill and affection between people.'' However, where Cicero becomes overly ambitious 
is that he adds the qualifier ``\ldots about all things.'' This is the fundamental disconnect between Cicero’s 
reality and our own. There are \emph{no} two people in modern society who agree on all things. Fortunately, this 
disconnect can be repaired. I argue that we do not need agreement on all things to be friends. In fact, some 
of my closest friends are those with whom I have the strongest disagreements. However, we are able to maintain our 
friendship because we still have an agreed upon idea of virtue, however that is not the same definition of 
virtue that Cicero or the Romans might use. While Cicero may define virtue in the context of common natural 
law, in modern society every individual may have a different definition of virtue, but that does not mean there 
can be no agreement on that definition between any two individuals. 

\newpage
\printbibliography

\end{document}



