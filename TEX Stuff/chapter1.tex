\chapter{Foreward}

We have all wondered what it means to exist. What are we? Who are we? What other rhetorical questions
can I ask? It was on a chill March day in the two thousandth and twentieth year following the birth of Christ
that I found myself in solitude along a lake. It's often these moments of solitude that lead the most riveting
thoughts. When walking alone in the woods beside a lake you are reminded of how you are, after all, just a 
meatbag with a consciousness wobbling along in space.

When coming to this realization, many (or rather most) humans turn to religion to give them meaning. While
an understandable decision, I don't believe it is a satisfying answer. After all, there are dozens of religions
in the world, and each one of them is so confident that they are the right religion that they go to war over
it.

Alas, this book is not meant to bash religion by any means. In fact, I am religious myself. The point of
this book is to instead offer a different, more logical and philosphical warrant for valuing existance, and
why without first understanding existance we cannot truly value anything.

\pagebreak

After all, there are three possible scenerios we could have ended up in. Only one of them is a reality:

\begin{enumerate}
    \item We don't exist, life doesn't exist, and matter doesn't exist at all.
    \item We exist, but purely as clueless meatbags wandering about a cold and empty universe.
    \item We exist, and have consciousness, and are rational agents capable of getting a grip on reality.
\end{enumerate}

The first two presumptions are utterly useless. After all, if we do not exist, and are not conscious beings,
then we ought not even think of anything at all.

The latter of the three is of course the only one from which we can draw conclusions. Before we even begin
to talk about anything, we must first accept the latter of the three scenerios I put forth. Indeed, nothing
makes sense until we first actualize our existance. This is where this book begins.